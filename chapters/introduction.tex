Some basic ways to manipulate text are \textit{italics} and \textbf{bold}. One can reference Figures (see Figure \ref{fig:taltech} for an example) as well as cite references which are defined in the \textit{references.bib} file.\cite{spectre,example-reference} 

The \textit{Bibliography}, \textit{List of Figures} and \textit{List of Tables} are all automatically generated and references will be updated automatically as well. This means that if you've defined a citation but are not referencing it, it will not appear in the \textit{Bibliography}. This also means that any Figure / Table / Citations numbers are automatically updated as well. Numbering is done by order-of-appearance.

One can create an itemized list:
\begin{itemize}
    \item item a
    \item item b
    \item ...
\end{itemize}

Or enumerate them:
\begin{enumerate}
    \item item x
    \item item y
    \item ...
\end{enumerate}


\begin{figure}[ht]
    \centering
    \includegraphics[width=.5\textwidth]{figures/taltech.jpg}
    \caption{\textit{An image of the TalTech logo.}}
    \label{fig:taltech}
\end{figure}


A table with three columns can be seen in Table \ref{tab:requirements}.
\begin{longtable}{|p{0,5cm}|p{10cm}|p{3cm}|}
	\caption{\it{A table with some requirements}}
	\label{tab:requirements}\\ \hline
	\textbf{Nr} &  \textbf{Requirement} & \textbf{Weight}  \\
	\hline
	\endfirsthead
	\multicolumn{3}{l}%
	{\tablename\ \thetable\ -- \textit{Continues...}} \\
	\hline
	\textbf{Nr} &  \textbf{Requirement} & \textbf{Importance}  \\
	\hline
	\endhead
	\hline \multicolumn{3}{l}{\textit{Continues...}} \\
	\endfoot
	\hline
	\endlastfoot
1 & Price & High\\ \hline
2 & Variety& Middle\\ \hline
3 & Support& Low\\ \hline

\end{longtable}

We can use variables set in the \textit{main.tex} file to render values like our title (\doctitle) or supervisor names (\textbf{Supervisor}: \supervisor, \textbf{Co-supervisor}: \cosupervisor{}).